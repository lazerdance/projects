%        File: 2_theorie_wehr_dyballa_wehr_chen.tex
%     Created: Sa Mai 25 08:00  2019 C
% Last Change: Sa Mai 25 08:00  2019 C
%

\documentclass[a4paper]{article}
\usepackage[ngerman]{babel}
\usepackage[a4paper, left=2cm, right=3cm, top=2cm, bottom=2cm]{geometry}
\usepackage{listings}
\usepackage{tikz}
\usetikzlibrary{arrows}
\newenvironment{quelle}{\medskip \noindent\itshape Quelle: }{\bigskip}
\def\thesubsubsection{\alph{subsubsection})}
\newcommand{\hausaufgabenNr}{3}

\begin{document}
\setcounter{section}{\hausaufgabenNr}
\setcounter{subsection}{3}
\section*{Hausaufgabe \hausaufgabenNr}
Gruppenmitglieder:\\
Vincent Wehr\\
Yannek Wehr\\
Andreas Dyballa\\
Junyi Chen
\subsection{Scheduling-Theorie}
\subsubsection{} 
Die Verdrängung beschreibt das Anhalten bzw. Ersetzen eines Prozesses 
durch einen anderen Prozess nach der Schedulingstrategie.

\begin{quelle} de.wikipedia.org/wiki/Prozess\_(Informatik)\#Prozessumschaltung \end{quelle}

\subsubsection{}
Das HRRN-Verfahren bildet einen Kompromiss zwischen der Antwortzeit und der Fairness.

\begin{quelle}
  Folie 24
\end{quelle}

\subsubsection{}
Die Prioritätsinvertierung wird angewendet, wenn zwei, um ein notwendiges Betriebsmittel kongurierende, 
Prozesse sich blockieren. Dies tritt auf, wenn der betriebsmittelnutzende Prozess eine niedrigere Priorität als der Kongurierende hat, da der niedrig priorisierte Prozess die Ressource nicht abgeben kann.

Dieser Effekt ensteht durch die Existenz dritter Prozesse mit einer Priorität zwischen den Prozessen.
Denn dieser verhindert die Ausführung des niedrigsten Prozesses und damit die Auflösung der Blockierung der Systemressource, da dieser beim Abgeben des höchsten Prozesses anstelle des Blockierenden ausgeführt werden würde.

Um diesem Effekt entgegen zu wirken, wird dem blockierenden Prozess die Priorität des wartenden Prozesses zugeordnet. Nach Freigabe des Betriebsmittels, bzw. Auflösen des Konfliktes, erhält er seine ursprüngliche Priorität.  
Dies beschreibt die Prioritätsinvertierung.

Ohne sie wird ein hoch priorisierter Prozess in dem beschriebenen Szenario dauerblockiert. Dies führte z.B. fast zu dem Verlust der Pathfinder-Marssonde.

\begin{quelle}
  
  Folie 21
  
  de.wikipedia.org/wiki/Prioritätsinversion
\end{quelle}

\subsubsection{}
Beim Offline Scheduling sind bereits alle Prozesse bekannt, besonders ihre Ankünfte. Wodurch bereits
vor der Ausführungszeit der Ablaufplan erstellt werden kann. Wohingegen beim Online Scheduling nur während
der Laufzeit die aktuellen Prozesse bekannt, nicht aber die Zukünftigen.
Dementsprechend wird die Entscheidungsfindung auf Grund von 
unvollständiger Informationen abgehandelt.

Somit wird beim Offline Scheduling ein totales Wissen über alle Prozesse benötigt und beim Online Scheduling kein Vorwissen benötigt.

\begin{quelle}
  Folie 7
\end{quelle}

\subsubsection{}
Das Hard real-time system und das Soft real-time system sind beide Echtzeitsysteme und müssen somit
Ergebnisse somit in einer gegebenen Zeit antworten muss. Das Hard real-time system toleriert
Antwortzeitüberschreitungen nicht. Wohingegen Soft real-time systeme die vorgegebene Antwortzeit als Richtwert in Form eines Mittelwertesa oder ähnliches betrachten.

\begin{quelle}

  Folie 32
  
  de.wikipedia.org/wiki/Echtzeitsystem

\end{quelle}

\end{document}
