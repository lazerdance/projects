%        File: 2_theorie_wehr_dyballa_wehr_chen.tex
%     Created: Sa Mai 25 08:00  2019 C
% Last Change: Sa Mai 25 08:00  2019 C
%

\documentclass[a4paper]{article}
\usepackage[ngerman]{babel}
\usepackage[a4paper, left=2cm, right=3cm, top=2cm, bottom=2cm]{geometry}
\usepackage{listings}
\usepackage{pgfgantt}
\newenvironment{quelle}{\medskip \noindent\itshape Quelle: }{\medskip}
\def\thesubsubsection{\alph{subsubsection})}
\newcommand{\hausaufgabenNr}{4}

\begin{document}
\setcounter{section}{\hausaufgabenNr}
\setcounter{subsection}{3}
\section*{Hausaufgabe \hausaufgabenNr}
Gruppenmitglieder:\\
Vincent Wehr\\
Yannek Wehr\\
Andreas Dyballa\\
Junyi Chen
\subsection{Synchronisation/Kooperation}
\subsubsection{}
Die Threads unterbrechen sich gegenseitig auch an unerwünschten Stellen, da der Threadwechsel 
nicht kontrolliert ist. Zum Beispiel kann das Laufband weiter gehen, während der Schraubarm noch am 
Produkt ist und somit das Auto zerkratzen.

\subsubsection{}
Spurious Wakeup ist das unberechtigte Aufwecken eines Threads durch das Betriebssystem, das 
selbst ohne das Senden des expliziten erwarteten Signals passieren kann. Um das zu vermeiden, wird 
nach dem Aufwecken in einer Schleife um den Waitbefehl überprüft, ob die Bedingung erfüllt ist. 
Wenn dies nicht der Fall ist wird der Thread wieder pausiert.

\begin{quelle}
  Vorlesung 4, Folie 50
\end{quelle}

\subsubsection{}
\begin{lstlisting}
State s = belt; 			// Werte: 'belt' oder 'screw'
Signal schraubenFertig;
Signal bandFertig;
Mutex nichtZerkratzen;

Befoerderungsband-Thread: 
while(1) { 
	while(s == screw) {
		wait(schraubenFertig);
	}
  	lock(nichtZerkratzen);
	belt_move(1);     		// Band nach vorne bewegen
	s = screw;
	unlock(nichtZerkratzen);
	signal(bandFertig);
}

Schraubarm-Thread: 
while(1) { 
	while(s == belt) {
    		wait(bandFertig);
	}
	lock(nichtZerkratzen);
	arm_dock();         		// Schraubmaschine andocken 
	screw();                	// Schrauben festdrehen
	arm_undock();   		// Schraubmaschine abdocken
	s = belt;
	unlock(nichtZerkratzen);
	signal(schraubenFertig);
}
\end{lstlisting}
\newpage

\subsubsection{}
Es gibt zwei Arten der expliziten Prozess-/Threadinteraktionen: Konkurrenz und Kooperation.
Es handelt sich hier um eine Konkurrenz, da die beiden Threads um das Betriebsmittel, die 
Autos, exklusiv konkurrieren. Das bedeutet, dass entweder der Schraubarm-thread oder der 
Band-thread mit dem Auto interagieren kann, aber nie beide gleichzeitig.

\begin{quelle}
  Vorlesung 4, Folie 5
\end{quelle}

\subsection{Periodische Prozesse}
\subsubsection{}
Für jeden Prozess gilt $0 \; < \; b_i \le d_i \le p_i$:  
\\
\[ A: 0 \; < \; 1 \le 6 \le 6 \]
\[ B: 0 \; < \; 2 \le 6 \le 6 \]
\[ C: 0 \; < \; 1 \le 4 \le 6 \]
Des weiteren gilt $\sum \limits_{i} \frac{b_i}{p_i} \; \le \; 1$:
\[ \frac{1}{6} + \frac{2}{6} + \frac{1}{4} = \frac{3}{4} \;\le\; 1 \]
Damit wird das notwendige Kriterium erfüllt und ein zulässiger Schedule ist möglich. Laut c) existiert z.B. das RMS.

\begin{quelle}
  Vorlesung 4, Folie 37-41
\end{quelle}

\subsubsection{}
\begin{ganttchart}[
    hgrid=true,
    vgrid={*1{dotted}}
]{1}{28}
\gantttitlelist{1,...,28}{1} \\
\ganttgroup{Hyperperiode}{1}{12}\\
\ganttbar{Task A}{2}{2} 
\ganttbar{}{7}{7}
\ganttbar{}{14}{14}
\ganttbar{}{19}{19}
\ganttbar{}{26}{26}\\
\ganttbar{Task B}{3}{4}
\ganttbar{}{8}{9}
\ganttbar{}{15}{16}
\ganttbar{}{20}{21}
\ganttbar{}{27}{28}\\
\ganttbar{Task C}{1}{1} 
\ganttbar{}{5}{5}
\ganttbar{}{10}{10}
\ganttbar{}{13}{13}
\ganttbar{}{17}{17}
\ganttbar{}{22}{22}
\ganttbar{}{25}{25}
\ganttvrule{}{5}
\ganttvrule{}{6}
\ganttvrule{}{10}
\ganttvrule{}{12}
\ganttvrule{}{17}
\ganttvrule{}{18}
\ganttvrule{}{22}
\ganttvrule{}{24}
\end{ganttchart}

\subsubsection{}
Ja das RMS ist ein gültiger Schedule, da laut a) das notwendige Kriterium und hiermit das hinreichende Kriterium erfüllt wird:

\[ \sum \limits_{i=1}^n \frac{b_i}{p_i} \;\le\; n (2^{\frac{1}{n}} - 1) \]
\[ \frac{1}{6} + \frac{2}{6} + \frac{1}{4} = \frac{3}{4} \; = \; 0.75 \]

\[ 0.75 \; \le \; 0.75682\cdots\]

\begin{quelle}
  Vorlesung 4, Folie 38
\end{quelle}

\subsubsection{}
Wird ein weiterer Prozess D mit $(D=3, P=19)$ hinzugefügt, kann das hinreichende Kriterium nicht mehr erfüllt werden.
\[ \frac{1}{6} + \frac{2}{6} + \frac{1}{4} + \frac{3}{19} \approx 0.907894 \; \not\leq \; 0.75682\cdots\]

Da $0.907894\cdots \le 1$, ist allerdings das notwendige Kriterium erfüllt und somit ist ein anderes Scheduling als das RMS möglich. 

\begin{quelle}
  Vorlesung 4, Folie 37-41
\end{quelle}

\end{document}
