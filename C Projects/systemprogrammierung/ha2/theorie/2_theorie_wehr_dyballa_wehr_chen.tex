%        File: 2_theorie_wehr_dyballa_wehr_chen.tex
%     Created: Sa Mai 25 08:00  2019 C
% Last Change: Sa Mai 25 08:00  2019 C
%
\documentclass[a4paper]{article}
\usepackage[ngerman]{babel}
\usepackage[a4paper, left=2cm, right=3cm, top=2cm, bottom=2cm]{geometry}
\usepackage{listings}
\usepackage{tikz}
\usepackage{tabularx}
\usetikzlibrary{arrows}
\newenvironment{quelle}{\medskip \noindent\itshape Quelle: }{\bigskip}
\def\thesubsubsection{\alph{subsubsection})}

\begin{document}
\setcounter{section}{2}
\setcounter{subsection}{2}
\section*{Hausaufgabe 2}
Gruppenmitglieder:\\
Vincent Wehr\\
Yannek Wehr\\
Andreas Dyballa\\
Junyi Chen
\subsection{}
\subsubsection{}
Ein Prozess ist eine Instanz eines Programmes. Er ist spezialisiert auf deren Ausführung.
\begin{itemize}
  \item besitzt Ressourcen
  \item ist eine Verarbeitungsvorschrift
  \item Aktivitätsträger
\end{itemize}
Durch Prozesse können die Ausführungen verschiedener Programme seperiert werden. Des weiteren können durch 
diese Trennung logische Abhängigkeiten dargestellt und aufgelöst werden.

\begin{quelle} 
  Folie 2,3 
\end{quelle}

\subsubsection{}
\begin{tabular}{l l}
  Nebenläufigkeit: & Logisch simultane Verabeitung von Operationsströmen\\
  & (wird durch Sequentialisierung gelöst) \\ \\
  Parallelität: & Tatsächlich simultane Verabeitung von Operationsströmen z.B. auf mehrern Prozessoren
\end{tabular}

\begin{quelle}
  Folie 12, 13
\end{quelle}

\subsubsection{}
\begin{tabular}{ll}
  Explizietes Umschalten: 	& Prozess gibt die CPU ab \\
  Implizietes Umschalten:	& Prozess löst einen blockierenden Systemaufruf aus
\end{tabular}

\begin{quelle}
  Folie 17
\end{quelle}

Bei einer Prozessumschaltung werden die Registerinhalte in den PCB gespeichert und der Prozesszustand wird aktualisiert.
Danach wird ein anderer Prozess geladen. Dieser erhält sein passenden virtuellen Adressraum aus dem PCB, somit werden 
seine Registerinhalte geladen und der Prozess wird, durch den gespeicherten Befehlszähler, an der Stelle fortgesetzt wo dieser
unterbrochen wurde.

\begin{quelle}
  Folie 18
\end{quelle}

\subsubsection{}
\begin{itemize}
  \item Threads sind der ausführende Teil eines Prozesses, in Form eines Kontrollflusses des Prozesses.
  \item Prozesse benutzen seperate virtuelle Adressräume und können sich somit nicht beeinflussen.
  \item Threads benutzen den selben virtuellen Adressraum.
  \item Beim Umschalten wird der Ausführungszustand eines Prozesses im PCB gespeichert, bei Prozessen hingegen in einer Threadtabelle.
  \item Threads werden von dem startenden Prozess verwaltet, Prozesse vom Kernel.
\end{itemize}
\begin{quelle}
  Folie 25, 26, 33, 34 
\end{quelle}
%\break
\subsection{}
\subsubsection{}
\begin{tikzpicture}[->,>=stealth',shorten >=1pt,auto,node distance=1.7cm,
                    thick,main node/.style={circle,draw,font=\sffamily\Large\bfseries}]
		    
		    \node[main node] (1) {A};
		    \node[main node] (2) [above right of=1] {B};
		    \node[main node] (3) [below right of=1] {C};
		    \node[main node] (4) [right of=3] {D};
		    \node[main node] (6) [below of=3] {F};
		    \node[main node] (5) [left of=6] {E};
		    \node[main node] (7) [right of=6] {G};
		    \node[main node] (8) [below of=5] {H};
		    \node[main node] (9) [right of=7] {I};
		    \node[main node] (10) [above right of=4] {J};
		    \node[main node] (11) [above right of=9] {K};

		    \path[every node/.style={font=\sffamily\small}]
		    (1) edge  	     (2)
		        edge         (3)
		    (3) edge         (4)
		        edge         (6)
		    (5) edge         (6)
		    (4) edge         (7)
		    (6) edge         (7)
		    (2) edge         (10)
		    (7) edge         (9)
		    (9) edge         (11)
		    (10)edge         (11)
		    (7) edge [bend right ]  (10);
  \end{tikzpicture}
\subsubsection{}
\begin{lstlisting}
fork k:=jobH()
fork e:=jobE()
     a:=jobA()
fork b:=jobB(a)
     c:=jobC(a)
join e:=jobE()
fork f:=jobF(c,e)
     d:=jobD(c)
join f:=jobE(c,e)
     g:=jobG(d,f)
join b:=jobB(a)
fork j:=jobJ(b,g)
     i:=jobI(g)
join j:=jobJ(b,g)
     k:=jobK(i,j)
join k:=jobH()
\end{lstlisting}


\end{document}


